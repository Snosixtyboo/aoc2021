This day was a trivial parsing task, nothing interesting about the runtime (it's linear). However, I started getting more serious about exploiting the type checking capabilities of Typescript. I have now enabled the $\texttt{strict}$ constraint on $\texttt{tsc}$. It is interesting how this almost automatically identifies things that can logically go wrong in your program and forces you to handle them. In order to avoid/catch these issues and fail gracefully, the code has become more verbose and I have started to include more advanced TypeScript features, like type guards and type unions. It might be a little overkill and not the optimal design for this problem, but the learning effect is appreciated.