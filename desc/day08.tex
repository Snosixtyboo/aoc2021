Nothing special today, the solution is more of a logic puzzle applied to the display of numbers. Given the set of available numbers, we can establish rules to derive, based on the illuminated segments, which segment combination corresponds to which number and work our way up from there. I used the fact that the numbers that are easy to identify, $1,4,7,8$, have relevant features (upper left corner, lower left corner, right-hand side filled) to filter and find the remaining numbers. A more elegant solution, as seen online, would have been to notice that we can create a histogram of how often each segment lights up in all observed numbers. For each number, the sum of the histogram counts that correspond to its segments is unique. This is a generalization of what allowed us to identify the numbers $1,4,7,8$ in the first place. Awesome, but didn't think of it myself. 🎅
<br><br>
Run time is not that interesting here, as we can't skip lines and within a line, everything is constant.