Today was math fundamentals, but still fun for me to derive the respective solutions. For task 1, we can use a somewhat trivial argumentation: Consider the target candidate $x$ to initially lie at $x_0$ somewhere within the field of crabs so that their accumulated distance to $x_0$ is $d_{x_0}$. With this initial value, we can split the crabs into two groups of different size: $L_0$ crabs on the left, and $R_0$ on the right of $x_0$. Moving $x$ one unit to the left will reduce the distance to travel for all $L_0$ crabs by $1$, and increase it by 1 for the other $R_0$ crabs: $d_{x_{0}-1} = d_{x_0}+(R_0-L_0)$. As long as the groups stay the same, the distance just goes up by the difference $(R_0-L_0)$. But if we move past a crab, say, on the left, it now switches from the left to the right group. With each additional step, the total distance to $x$ now increases by $R_0+1$ and decreases by $L_0-1$. The same, but mirrored, is true for moving one step to the right, where we have $d_{x_{0}+1} = d_{x_0}+(L_0-R_0)$. The best location $x^\ast$ to keep $d_{x^\ast}$ as low as possible is thus in a spot where $R = L$: regardless of the exact value for $d_{x^\ast}$, from there, no matter where we move, things can only get worse. If we move left, we eventually get $R > L$ and then $d_{x - 1} = d_{x}+(R-L)$ will grow from there. If we instead move right, we eventually get $L > R$ and $d_{x+1} = d_{x}+(L-R)$ will grow. Hence, the solution for the first task is the median.
<br><br>
For the second solution, I derived it formally. First of all, Gauß tells us that $\sum_{d=1}^D = \frac{D^2 +D}{2}$, so given a particular target $x^\ast$, we could compute each crab's fuel expense for moving there from $x$ as $\frac{(x^\ast-x)^2 + |x^\ast-x|}{2}$. Now to find the ideal value $x^\ast$, we want to choose it such that it minimizes the sum of this term over all $N$ crabs, i.e., $\underset{x^\ast}{\text{min}} \sum_i^N \frac{(x^\ast-x_i)^2 + |x^\ast-x_i|}{2}$. 
<br><br>
We can do this by taking the first derivative w.r.t. $x^\ast$ and setting it to 0. If we do this using fundamental differentiation rules and cancelling, we eventually get $\sum_i^N x^\ast - x_i + \frac{[-1,1]}{2} = 0$. The trickiest part here is probably the $[-1,1]$ interval: the absolute value has no universal derivative. As this is a sum over multiple absolute values and a piece-wise smooth function, there are $N$ points where it is not differentiable. In the context of optimization, there is the concept of the "subderivative", which is a tool for solving such tasks in practice. For the absolute value, the subderivative is the sign function, which returns $1$ for positive inputs, $-1$ for negative and $0$ otherwise, with the exception of $[-1,1]$ at the origin. So we can bound the values that this part of the optimization problem will take on with $[-1,1]$. We can move part of this result to the right: $\sum_i^N x^\ast = \sum_i^N x_i - \frac{ [-1,1] }{2}$. Since $x^\ast$ doesn't change with $i$, we can replace its sum by multiplication and write $N x^\ast = \sum_i^N x_i - \frac{[-1,1]}{2}$. Dividing by $N$, we get $x^\ast = \frac{1}{N} \sum_i^N x_i - \frac{[-1,1]}{2}$. 
<br><br>
Now $\frac{1}{N}\sum_i^N x_i$ is just the mathematical definition of the mean value over all $x_i$ values, $\bar{x}$. So we get $x^\ast = \bar{x} - \sum_i^N \frac{[-1,1]}{2N}$. The optimum will be close to the mean, all that's left to figure out is a bound for $\sum_i^N \frac{[-1,1]}{2N}$. We can simplify the sum to $N\frac{[-1,1]}{2N}$. And since the sum is divided by $2N$, we can bound the entire expression by the range $[-\frac{1}{2},\frac{1}{2}]$. Since the mean computation will probably give us a floating point number, we thus have to check the possible integer values in the range $\lfloor \bar{x} \rceil \pm 1$, i.e., round the mean and test the result, as well as the next integer above and below.