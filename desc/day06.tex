This day taught me how much there is still out there to learn, specifically in the field of maths. The solution I came up with for $n$ days and $m$ initial fish runs in $\mathcal{O}(n^2 + m)$, of all things. The basic idea was to divide them into "litters" of the original school of fish. The first school will record how many fish it spawned each day, including those spawned by its litters. Every 7 days, there is a complete new litter who will keep track of its offspring by looking at the memoized offspring values of the original school. It works? But there are much more elegant solutions, e.g., one posted in Rust by Alexander Weinrauch <a href="https://github.com/AlexAUT/AoC2021/blob/master/day6/src/main.rs">here</a>. Even better, there seem to be advanced math-based solutions for computing much, much longer periods of time <a href="https://www.reddit.com/r/adventofcode/comments/ra3f5i/2021_day_6_part_3_day_googol/">here</a>. Humbling! That's what christmas is all about! :D