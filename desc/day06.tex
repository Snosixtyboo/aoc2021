This day taught me how much there is still out there to learn, specifically in the field of linear algebra. The solution I came up with for $n$ days and $m$ initial fish runs in $\mathcal{O}(n^2 + m)$, of all things. The basic idea was to divide them into "litters" of the original school of fish. The first school will record how many fish it spawned each day, including those spawned by its litters. Every 7 days, there is a complete new litter who will keep track of its offspring by looking at the memoized offspring values of the original school. It works? But a much more elegant solution would have been to collect the number of fish that have a particular time left until they spawn, as done, e.g., by Alexander Weinrauch in Rust <a href="https://github.com/AlexAUT/AoC2021/blob/master/day6/src/main.rs" target="_blank">here</a>. 
<br><br>
Even better, there seem to be advanced matrix-based solutions for computing much, much longer periods of time <a href="https://www.reddit.com/r/adventofcode/comments/ra3f5i/2021_day_6_part_3_day_googol/" target="_blank">here</a>. Upon further reading, I found out about <a href="https://en.wikipedia.org/wiki/Matrix_difference_equation" target="_blank">matrix difference equations</a>, of which I previously lacked any knowledge. Including the fact that they can fairly easily be used to compute (and derive) the closed-form solution of the Fibonacci numbers. Humb(ug)ling! That's what christmas time is all about! 🎄